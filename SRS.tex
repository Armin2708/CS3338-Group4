\documentclass{article}
\usepackage{graphicx}
\usepackage[utf8]{inputenc}
\usepackage{geometry}
\usepackage{fancyhdr}
\usepackage{hyperref}
\usepackage{lipsum}
\usepackage{tocloft}
\usepackage{array}

\title{CS3338_Final_Project_SRS}
\author{ryantakeshita }
\date{December 2024}

\begin{document}

\begin{titlepage}
   \begin{center}
       \vspace*{1cm}

       \textbf{Software Requirement Specification (SRS)}

       \vspace{0.5cm}
        QTC Smart Dashboard
            
       \vspace{1.5cm}

       \textbf{Leonardo Granados-Castillo\\Armine Rad\\Benjamin Saucedo\\Ryan Takeshita}

       \vfill
            
            
       \vspace{0.8cm}

       California State University, Los Angeles\\
       11 December 2024
            
   \end{center}
\end{titlepage}


\tableofcontents
\setcounter{}
\renewcommand{\cftsecfont}{\bfseries}
\renewcommand{\cftsubsecfont}{\itshape}

\pagebreak

\section{Versions Table}


\begin{tabular}{ | m{5em} | m{2cm}| m{7cm} | m{2cm} | } 
  \hline
  Name & Date & Reason for Update & Version\\ 
  \hline
  Ryan Takeshita & 11/21/24 & Starting the document. Laying out the system requirements as specified in Snapshot 1. & 1.0 \\ 
  \hline
  Ryan Takeshita & 11/25/24 & Added functional requirements for the Pagination feature of the Smart Dashboard. & 1.1\\ 
  \hline
  Ryan Takeshita & 12/08/24 & Updated the document to reflect the system requirements for the Side Navigation bar and the Help Button features. & 1.2\\ 
  \hline
  Ryan Takeshita & 12/09/24 & Added information to document that shows the related requirements for the Search Filtering feature being added to the Dashboard.& 1.3\\
  \hline
\end{tabular}
\pagebreak




\section{Introduction}
\subsection{Overview}
The project that we are creating is the Smart Dashboard in collaboration with the QTC company. QTC provides medical examination information and keeps information of these examinations stored in their Oracle/SQL databases. QTC has tasked us with helping them create a new user interface to see errors the system of these records. This will be done by creating a Smart Dashboard to streamline this process.
\subsection{Purpose}
This document has been created to help describe the system requirements specifications for this Smart Dashboard app. In this document, the external interface requirements will be given by explaining the Dashboard's User Interface as well as the Software Interfaces that will be used. At the same time, we will explore the Legal and Ethical considerations that must be made when working on this project. Finally, a glossary explaining terms and abbreviations used in the project will be appended to the end of this document.
\subsection{Intended Audience}
This SRS document is intended to be viewed by the service desk staff who will be using the Smart Dashboard. This document will lay out a guide for the User Interface to show what exactly they will be working with and how the Dashboard's components should work in aiding them in their jobs. At the same time, they can view the Software Interfaces to understand how this Dashboard runs and executes its tasks.


\section{External Interface Requirements}

\subsection{User Interface}
The User Interface will consist mainly of a window that displays the Smart Dashboard. The Dashboard will essentially show a table that lists the error reports that are retrieved from the database.\\
In addition to this, there will be buttons on the Dashboard that allow the transit between pages of different error records through Pagination. These buttons include:
\begin{itemize}
    \item Previous Page (takes user to previous page of records)
    \item Next Page (takes user to next page of records)
    \item '...' (to see intermediary pages)
    \item Final Page (takes user to final page of records)
    \item More Pages (loads in more pages of records)
\end{itemize}
There will also be a dropdown menu in the top left of the Dashboard that allows the user to select how many error report entries are listed per page.\\

Another aspect of the user interface will be a Side Navigation Bar that will be on the left side of the Smart Dashboard that will allow the user to easily navigate different data areas of the error reports as well as just improve the UI/UX overall.\\

An additional feature is the Help Button. This is a button located towards the bottom left of the Dashboard that will allow the user to find assistance in the form of guides on how the features of the Smart Dashboard work if they ever forget or need clarification.\\

The Dashboard also comes with a filtering ability for the user to specify which records they would like to show. With this filtering ability comes with the implementation of logging in as a certain suer with Windows Authentication. The user's role that they have upon login will give them access to certain filterings.

\subsection{Software Interfaces}
For the database access, the Smart Dashboard will communicate with the SQL/Oracle servers of QTC depending on the service requested.

The .NET framework is the basis for the functionality of the Smart Dashboard.

For Pagination, the backend functionality of the pages will be done through Razor Pages

The main interface that will be used for the UI/UX of the Smart Dashboard will be the Bootstrap 5 framework. Bootstrap 5 has several built-in modules that make buttons and other interactive features easy to implement.



\section{Legal and Ethical Considerations}

\subsection{Healthcare Related Database}
QTC's error database is ultimately one that is based in data from the health care field. For this reason, it is extremely important to take correct security measures when handling this data since it relates back to customers' sensitive information. To handle this, we must take advantage of proper encapsulation methods in our code as well as implementing user authorization methods for whoever uses the Smart Dashboard.

\subsection{Ensuring Database is Intact}
Because the Smart Dashboard is accessing QTC's database, we must ensure that the application itself does not affect the original data. Employing proper access control techniques will prevent users from altering or somehow affecting QTC's intellectual property.



\section{Glossary}
\subsection{Terms}
\begin{itemize}
    \item Dashboard: A visual menu that helps the suer understand some given information and work with/manipulate that information.
    \item Dropdown menu: A menu that when selected will reveal a sub-menu of several choices
    \item Pagination: the ability to have several pages of displayed data that can be traversed
    \item .NET: this is in reference to ASP.NET which is essentially a general web development framework that streamlines the ability to pull data from a source, do something with that data, and display it to the user while also letting them interact with it
    \item Razor Pages: a feature of .NET interface that allows for pages of data to easily be setup on a webpage
    \item Bootstrap 5: a framework that comes with built-in classes that allow to easily setup User Interface features for a website
\end{itemize}

\subsection{Abbreviations}
\begin{itemize}
    \item QTC: Quality. Timeless. Customer Service. (company)
    \item UI/UX: User Interface / User Experience
\end{itemize}



\end{document}
